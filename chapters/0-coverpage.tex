%-------TITLE PAGE------%
\begin{titlepage}
	\center
	\begin{tikzpicture}[overlay,remember picture]
		\draw [line width=3pt,rounded corners=0pt,]
		($ (current page.north west) + (25mm,-25mm) $)
		rectangle
		($ (current page.south east) + (-15mm,25mm) $);
		\draw [line width=1pt,rounded corners=0pt]
		($ (current page.north west) + (26.5mm,-26.5mm) $)
		rectangle
		($ (current page.south east) + (-16.5mm,26.5mm) $);
	\end{tikzpicture}
	
	{\large \bfseries ĐẠI HỌC QUỐC GIA HÀ NỘI\\ TRƯỜNG ĐẠI HỌC CÔNG NGHỆ}\\[1cm]
	\includesvg[width=0.25\linewidth]{images/Logo_HUET.svg}\\[1cm]
	{\Large  \bfseries Đoàn Duy Tùng}\\[1.5cm]
	{ \LARGE \bfseries  NGHIÊN CỨU GIẢI PHÁP }\\[0.2cm]
    {\LARGE \bfseries KIỂM THỬ ĐƠN VỊ TỰ ĐỘNG CHO MÃ NGUỒN C/C++}\\[0.2cm]
     { \LARGE \bfseries CHỨA HÀM THIẾU ĐỊNH NGHĨA}\\[0.2cm]
	\hfill\\[2cm]
	{\large \bfseries KHÓA LUẬN TỐT NGHIỆP ĐẠI HỌC HỆ CHÍNH QUY}\\	
	{\large \bfseries Ngành: Công nghệ thông tin}	
	\hfill\\[5.3cm]	
	{\large \bfseries HÀ NỘI - 2023}\\	
	\vfill
\end{titlepage}

%-------TITLE PAGE+6hbk,------%
\begin{titlepage}
	\center
	\begin{tikzpicture}[overlay,remember picture]
	\draw [line width=3pt,rounded corners=0pt,]
	($ (current page.north west) + (25mm,-25mm) $)
	rectangle
	($ (current page.south east) + (-15mm,25mm) $);
	\draw [line width=1pt,rounded corners=0pt]
	($ (current page.north west) + (26.5mm,-26.5mm) $)
	rectangle
	($ (current page.south east) + (-16.5mm,26.5mm) $);
	\end{tikzpicture}
	
	{\large \bfseries ĐẠI HỌC QUỐC GIA HÀ NỘI\\ TRƯỜNG ĐẠI HỌC CÔNG NGHỆ}\\[2cm]
% 	\includegraphics[width=0.25\linewidth]{images/Logo_UET.png}\\[1cm]
	
	{\Large  \bfseries Đoàn Duy Tùng}\\[2cm]		
	{ \LARGE \bfseries  NGHIÊN CỨU GIẢI PHÁP }\\[0.2cm]
    {\LARGE \bfseries KIỂM THỬ ĐƠN VỊ TỰ ĐỘNG CHO MÃ NGUỒN C/C++}\\[0.2cm]
     { \LARGE \bfseries CHỨA HÀM THIẾU ĐỊNH NGHĨA}\\[0.2cm]
	\hfill\\[1.5cm]
	{\large \bfseries KHÓA LUẬN TỐT NGHIỆP ĐẠI HỌC HỆ CHÍNH QUY}\\	
	{\large \bfseries Ngành: Công nghệ thông tin}
	\hfill\\[2cm]
	\begin{flushleft}
	    	{\large \bfseries Cán bộ hướng dẫn: PGS. TS. Phạm Ngọc Hùng}\\
	\end{flushleft}
	\hfill\\[2.5cm]	
	\begin{flushleft}
% 	{\large \bfseries Cán bộ đồng hướng dẫn: CN. Bùi Quang Cường}\\	
	\end{flushleft}
		\hfill\\[3cm]	
	{\large \bfseries HÀ NỘI - 2023}\\		
	\vfill		
\end{titlepage}

%-------TITLE PAGE+6hbk,------%
\begin{titlepage}
	\center
	\begin{tikzpicture}[overlay,remember picture]
	\draw [line width=3pt,rounded corners=0pt,]
	($ (current page.north west) + (25mm,-25mm) $)
	rectangle
	($ (current page.south east) + (-15mm,25mm) $);
	\draw [line width=1pt,rounded corners=0pt]
	($ (current page.north west) + (26.5mm,-26.5mm) $)
	rectangle
	($ (current page.south east) + (-16.5mm,26.5mm) $);
	\end{tikzpicture}
	
	{\large \bfseries VIETNAM NATIONAL UNIVERSITY, HANOI\\ UNIVERSITY OF ENGINEERING AND TECHNOLOGY}\\[2cm]
% 	\includegraphics[width=0.25\linewidth]{images/Logo_UET.png}\\[1cm]
	
	{\Large  \bfseries Doan Duy Tung}\\[2cm]	
	{ \LARGE \bfseries RESEARCH ON AN AUTOMATED UNIT TESTING}\\[0.2cm] 
	{\LARGE \bfseries SOLUTION FOR C/C++ SOURCE CODE }\\[0.2cm]
    {\LARGE \bfseries CONTAINING MISSING FUNCTION DEFINITIONS }\\[0.2cm]
	\hfill\\[1.5cm]
	{\large \bfseries BACHELOR’S THESIS}\\	
	{\large \bfseries Major: Information Technology}
	\hfill\\[2cm]
	\begin{flushleft}
	    	{\large \bfseries Supervisor: Assoc. Prof. Dr. Pham Ngoc Hung}\\
	\end{flushleft}
	\hfill\\[2.5cm]	
	\begin{flushleft}
% 	{\large \bfseries Cán bộ đồng hướng dẫn: CN. Bùi Quang Cường}\\	
	\end{flushleft}
		\hfill\\[3cm]	
	{\large \bfseries Hanoi - 2023}\\		
	\vfill		
\end{titlepage}

\changefontsizes[16pt]{13pt}
\addtocontents{toc}{\vspace{-1cm}}
\addcontentsline{toc}{chapter}{Lời cam đoan}
\begin{center}
    \textbf{LỜI CAM ĐOAN}
\end{center}

Em xin cam đoan: Khóa luận tốt nghiệp với đề tài “Nghiên cứu giải pháp kiểm thử đơn vị tự động cho mã nguồn C/C++ chứa hàm thiếu định nghĩa” trong báo cáo này là của em. Những gì em viết ra không có sự sao chép từ các tài liệu, không sử dụng kết quả của người khác mà không trích dẫn cụ thể. Đây là công trình nghiên cứu cá nhân em tự phát triển, không sao chép mã nguồn của người khác. Nếu vi phạm những điều trên, em xin chấp nhận tất cả những truy cứu về trách nhiệm theo quy định của Trường Đại học Công nghệ - ĐHQGHN.

\begin{flushright}
	\begin{varwidth}{\linewidth}\centering
		Hà Nội, ngày 4 tháng 12 năm 2023\\
		Sinh viên\\[2cm]
		Đoàn Duy Tùng
	\end{varwidth}
\end{flushright}

\newpage

\addcontentsline{toc}{chapter}{Lời cảm ơn}
\begin{center}
    \textbf{LỜI CẢM ƠN}
\end{center}

Lời đầu tiên cho phép em bày tỏ lòng biết ơn chân thành đến  Khoa Công nghệ Thông tin - Trường Đại học Công nghệ, ĐHQGHN, nơi em đã có cơ hội tiếp xúc với những kiến thức mới mẻ và được học hỏi từ những người giáo viên xuất sắc khác.

Em xin gửi lời cảm ơn chân thành nhất đến thầy Phạm Ngọc Hùng, người đã là nguồn động viên và hướng dẫn quý báu trong suốt thời gian em học tập, làm việc tại phòng nghiên cứu. Sự hướng dẫn tận tâm và sự hỗ trợ của thầy là động lực lớn giúp em vượt qua những thách thức trong hành trình tìm hiểu, nghiên cứu và hoàn thiện khóa luận tốt nghiệp.

Cuối cùng, em xin được gửi lời cảm ơn chân thành tới anh Nguyễn Tùng Lâm, anh Nguyễn Vũ Bình Dương cũng như toàn thể các anh chị và các bạn tại Phòng thí nghiệm đảm bảo chất lượng phần mềm (Khoa Công nghệ thông tin, Trường Đại học Công nghệ, ĐHQGHN) đã luôn ủng hộ, động viên em trong quá trình hoàn thành khóa luận. Đặc biệt, em xin chân thành cảm ơn anh Trần Trọng Năm và các anh chị tại đơn vị FPT-GAM đã giúp đỡ, cho phép em sử dụng tài nguyên của đơn vị trong quá trình làm khóa luận.

Chúc mọi người luôn luôn vui vẻ và gặt hái được nhiều thành công trong cuộc sống.

\newpage
\addcontentsline{toc}{chapter}{Tóm tắt}
\begin{center}
    \textbf{TÓM TẮT}
\end{center}
\changefontsizes[16pt]{12pt}
\textit{\textbf{Tóm tắt: }} 
Kiểm thử các dự án nhúng C/C++ là bài toán nhận được sự quan tâm của các công ty phát triển phần mềm cũng như cộng đồng nghiên cứu trên thế giới. Nhằm phát hiện sớm các lỗi tiềm ẩn và nâng cao độ tin cậy của mã nguồn, quá trình kiểm thử thường được triển khai từ sớm, khi mã nguồn chưa đầy đủ. Để giảm chi phí, thời gian và công sức trong việc kiểm thử đơn vị, các phương pháp kiểm thử tự động đã và đang được áp dụng rộng rãi. Kiểm thử phần mềm tự động bao gồm hai hướng tiếp cận chính là kiểm thử tĩnh và kiểm thử động. Kế thừa hai hướng tiếp cận trên, kiểm thử tượng trưng động được đề xuất với nhiều cải tiến và thu được những thành tựu nhất định. Tuy nhiên, khi áp dụng các hướng tiếp cận này, một số hạn chế phát sinh do mã nguồn có thể chứa hàm thiếu định nghĩa. Khóa luận đề xuất một phương pháp nhằm tự động phát hiện và xử lý các hàm thiếu định nghĩa. Ý tưởng chính để xử lý hàm thiếu định nghĩa là sử dụng trình biên dịch và đồ thị cấu trúc mã nguồn để tìm kiếm các hàm thiếu định nghĩa rồi sinh thân hàm giả cho chúng. Khóa luận cũng đề xuất một phương pháp để xử lý lời gọi phương thức, giúp tăng độ phủ kiểm thử các đơn vị chứa nhiều tương tác giữa các đối tượng. Phương pháp đề xuất đã được cài đặt và tích hợp vào công cụ AKAUTAUTO - sản phẩm hợp tác nghiên cứu giữa Phòng thí nghiệm đảm bảo chất lượng phần mềm (khoa Công nghệ thông tin, Trường Đại học Công nghệ, ĐHQGHN) và đơn vị FPT-GAM (FPT Global Automative \& Manufactoring) để tiến hành thực nghiệm đánh giá tính hiệu quả của phương pháp. Kết quả thực nghiệm cho thấy rằng phương pháp có khả năng xử lý các hàm thiếu định nghĩa trong thời gian ngắn và có khả năng sinh dữ liệu kiểm thử tự động với độ phủ cao trên một số mã nguồn mở và mã nguồn thực tế thuộc đơn vị FPT-GAM. Những kết quả tích cực của phương pháp đề xuất cho thấy tiềm năng ứng dụng thực tế để kiểm thử đơn vị các dự án C/C++ song song với quá trình phát triển.

\vspace{-0.5cm}
\begin{flushleft}
  \textit{\textbf{Từ khóa: } Kiểm thử tự động, kiểm thử tượng trưng động, kiểm thử dòng điều khiển, hàm thiếu định nghĩa, sinh giả lập mã nguồn tự động, lời gọi phương thức}
\end{flushleft}

\newpage
\addcontentsline{toc}{chapter}{Abstract}
\begin{center}
    \textbf{ABSTRACT}
\end{center}
\changefontsizes[16pt]{12pt}
\textit{\textbf{Abstract: }} 
Testing embedded C/C++ projects is a challenge that has drawn the attention of software development companies as well as the research community worldwide. In order to detect hidden errors early and enhance the reliability of the source code, the testing process is often deployed early, when the source code is not yet complete. To reduce costs, time, and effort in unit testing, automated testing methods have been widely applied. Automated software testing comprises two main approaches: static testing and dynamic testing. Inheriting from these two approaches, concolic testing is proposed with various improvements and has achieved certain achievements. However, when applying these approaches, some limitations arise due to the possibility that the source code may contain undefined functions. This thesis proposes a method to automatically detect and handle undefined functions. The main idea for handling undefined functions is to use the compiler and the source code structure graph to search for undefined functions and then generate dummy functions for them. The thesis also proposes a method to handle method calls, helping to increase test coverage for units with multiple interactions between objects. The proposed method has been implemented and integrated into the AKAUTAUTO tool, a collaborative research product between the Software Quality Assurance Laboratory (Faculty of Information Technology, University of Engineering and Technology, VNU) and FPT-GAM (FPT Global Automotive \& Manufacturing) to conduct experiments evaluating the effectiveness of the method. Experimental results show that the method can handle undefined functions in a short time and can generate automatic test data with high coverage on some open-source code and actual source code at FPT-GAM. The positive results of the proposed method demonstrate its potential for practical application in unit testing for parallel C/C++ projects alongside the development process.

\vspace{-0.5cm}
\begin{flushleft}
  \textit{\textbf{Keywords: } Automated test data generation, concolic testing, control flow testing, function prototype, unimplemented function, automated stub generation, method call}
\end{flushleft}
\changefontsizes[16pt]{13pt}
