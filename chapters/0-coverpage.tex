%-------TITLE PAGE------%
\begin{titlepage}
	\center
	\begin{tikzpicture}[overlay,remember picture]
		\draw [line width=3pt,rounded corners=0pt,]
		($ (current page.north west) + (25mm,-25mm) $)
		rectangle
		($ (current page.south east) + (-15mm,25mm) $);
		\draw [line width=1pt,rounded corners=0pt]
		($ (current page.north west) + (26.5mm,-26.5mm) $)
		rectangle
		($ (current page.south east) + (-16.5mm,26.5mm) $);
	\end{tikzpicture}
	
	{\large \bfseries ĐẠI HỌC QUỐC GIA HÀ NỘI\\ TRƯỜNG ĐẠI HỌC CÔNG NGHỆ}\\[1cm]
	\includesvg[width=0.25\linewidth]{images/Logo_HUET.svg}\\[1cm]
	{\Large  \bfseries Đoàn Duy Tùng}\\[1.5cm]
	{ \LARGE \bfseries  NGHIÊN CỨU GIẢI PHÁP }\\[0.2cm]
    {\LARGE \bfseries KIỂM THỬ ĐƠN VỊ TỰ ĐỘNG CHO MÃ NGUỒN C/C++}\\[0.2cm]
     { \LARGE \bfseries CHỨA HÀM THIẾU ĐỊNH NGHĨA}\\[0.2cm]
	\hfill\\[2cm]
	{\large \bfseries KHÓA LUẬN TỐT NGHIỆP ĐẠI HỌC HỆ CHÍNH QUY}\\	
	{\large \bfseries Ngành: Công nghệ thông tin}	
	\hfill\\[5.3cm]	
	{\large \bfseries HÀ NỘI - 2023}\\	
	\vfill
\end{titlepage}

%-------TITLE PAGE+6hbk,------%
\begin{titlepage}
	\center
	\begin{tikzpicture}[overlay,remember picture]
	\draw [line width=3pt,rounded corners=0pt,]
	($ (current page.north west) + (25mm,-25mm) $)
	rectangle
	($ (current page.south east) + (-15mm,25mm) $);
	\draw [line width=1pt,rounded corners=0pt]
	($ (current page.north west) + (26.5mm,-26.5mm) $)
	rectangle
	($ (current page.south east) + (-16.5mm,26.5mm) $);
	\end{tikzpicture}
	
	{\large \bfseries ĐẠI HỌC QUỐC GIA HÀ NỘI\\ TRƯỜNG ĐẠI HỌC CÔNG NGHỆ}\\[2cm]
% 	\includegraphics[width=0.25\linewidth]{images/Logo_UET.png}\\[1cm]
	
	{\Large  \bfseries Đoàn Duy Tùng}\\[2cm]		
	{ \LARGE \bfseries  NGHIÊN CỨU GIẢI PHÁP }\\[0.2cm]
    {\LARGE \bfseries KIỂM THỬ ĐƠN VỊ TỰ ĐỘNG CHO MÃ NGUỒN C/C++}\\[0.2cm]
     { \LARGE \bfseries CHỨA HÀM THIẾU ĐỊNH NGHĨA}\\[0.2cm]
	\hfill\\[1.5cm]
	{\large \bfseries KHÓA LUẬN TỐT NGHIỆP ĐẠI HỌC HỆ CHÍNH QUY}\\	
	{\large \bfseries Ngành: Công nghệ thông tin}
	\hfill\\[2cm]
	\begin{flushleft}
	    	{\large \bfseries Cán bộ hướng dẫn: PGS. TS. Phạm Ngọc Hùng}\\
	\end{flushleft}
	\hfill\\[2.5cm]	
	\begin{flushleft}
% 	{\large \bfseries Cán bộ đồng hướng dẫn: CN. Bùi Quang Cường}\\	
	\end{flushleft}
		\hfill\\[3cm]	
	{\large \bfseries HÀ NỘI - 2023}\\		
	\vfill		
\end{titlepage}

%-------TITLE PAGE+6hbk,------%
\begin{titlepage}
	\center
	\begin{tikzpicture}[overlay,remember picture]
	\draw [line width=3pt,rounded corners=0pt,]
	($ (current page.north west) + (25mm,-25mm) $)
	rectangle
	($ (current page.south east) + (-15mm,25mm) $);
	\draw [line width=1pt,rounded corners=0pt]
	($ (current page.north west) + (26.5mm,-26.5mm) $)
	rectangle
	($ (current page.south east) + (-16.5mm,26.5mm) $);
	\end{tikzpicture}
	
	{\large \bfseries VIETNAM NATIONAL UNIVERSITY, HANOI\\ UNIVERSITY OF ENGINEERING AND TECHNOLOGY}\\[2cm]
% 	\includegraphics[width=0.25\linewidth]{images/Logo_UET.png}\\[1cm]
	
	{\Large  \bfseries Doan Duy Tung}\\[2cm]	
	{ \LARGE \bfseries RESEARCH ON AN AUTOMATED UNIT TESTING SOLUTION FOR C/C++ SOURCE CODE }\\[0.2cm]
    {\LARGE \bfseries CONTAINING MISSING FUNCTION DEFINITIONS }\\[0.2cm]
	\hfill\\[1.5cm]
	{\large \bfseries BACHELOR’S THESIS}\\	
	{\large \bfseries Major: Information Technology}
	\hfill\\[2cm]
	\begin{flushleft}
	    	{\large \bfseries Supervisor: Assoc. Prof., Dr. Pham Ngoc Hung}\\
	\end{flushleft}
	\hfill\\[2.5cm]	
	\begin{flushleft}
% 	{\large \bfseries Cán bộ đồng hướng dẫn: CN. Bùi Quang Cường}\\	
	\end{flushleft}
		\hfill\\[3cm]	
	{\large \bfseries Hanoi - 2023}\\		
	\vfill		
\end{titlepage}

\changefontsizes[16pt]{13pt}
\addtocontents{toc}{\vspace{-1cm}}
\addcontentsline{toc}{chapter}{Lời cam đoan}
\begin{center}
    \textbf{LỜI CAM ĐOAN}
\end{center}

Em xin cam đoan: Khóa luận tốt nghiệp với đề tài “Một số cải tiến phương pháp sinh dữ liệu Kiểm thử cho các dự án C/C++ dựa trên phân tích mã nguồn” trong báo cáo này là của em. Những gì em viết ra không có sự sao chép từ các tài liệu, không sử dụng kết quả của người khác mà không trích dẫn cụ thể. Đây là công trình nghiên cứu cá nhân em tự phát triển, không sao chép mã nguồn của người khác. Nếu vi phạm những điều trên, em xin chấp nhận tất cả những truy cứu về trách nhiệm theo quy định của Trường Đại học Công nghệ - ĐHQG Hà Nội.

\begin{flushright}
	\begin{varwidth}{\linewidth}\centering
		Hà Nội, ngày 31 tháng 05 năm 2021\\
		Sinh viên\\[2cm]
		Nguyễn Tùng Lâm
	\end{varwidth}
\end{flushright}

\newpage

\addcontentsline{toc}{chapter}{Lời cảm ơn}
\begin{center}
    \textbf{LỜI CẢM ƠN}
\end{center}

Lời đầu tiên cho phép em được gửi lời cảm ơn tới Khoa Công Nghệ Thông Tin – Trường Đại học Công nghệ - ĐHQG Hà Nội đã tạo điều kiện thuận lợi cho em được học tập, nghiên cứu và thực hiện đề tài tốt nghiệp này.

Em cũng xin được bày tỏ lòng biết ơn sâu sắc tới thầy Phạm Ngọc Hùng và thầy Võ Đình Hiếu đã tận tình hướng dẫn, đóng góp những ý kiến xác đáng để em có thể hoàn thành khóa luận một cách tốt nhất.

Em cũng vô cùng biết ơn những thầy cô trong trường đã tận tình giảng dạy, trang bị cho em những kiến thức quan trọng để em có hành trang vững chắc cho con đường học vấn của mình.

Cuối cùng, em xin được gửi lời cảm ơn chân thành tới anh Nguyễn Đức Anh, anh Trần Hoàng Việt cũng như toàn thể các anh chị và các bạn tại Phòng thí nghiệm đảm bảo chất lượng phần mềm (Khoa Công nghệ thông tin, Trường Đại học Công nghệ, ĐHQGHN) đã luôn ủng hộ, động viên em trong quá trình hoàn thành khóa luận.

Chúc mọi người luôn luôn vui vẻ và gặt hái được nhiều thành công trong cuộc sống.

\newpage
\addcontentsline{toc}{chapter}{Tóm tắt}
\begin{center}
    \textbf{TÓM TẮT}
\end{center}
\changefontsizes[16pt]{12pt}
\textit{\textbf{Tóm tắt: }} 
Kiểm thử đơn vị cho các dự án C/C++, đặc biệt là các thư viện và dự án nhúng, đã và đang trở thành một bài toán lớn. Nhằm nâng cao độ tin cậy của mã nguồn, các phương pháp kiểm thử phần mềm tự động đang nhận được đông đảo sự quan tâm và đang được áp dụng rộng rãi trong cả cộng đồng nghiên cứu lẫn các công ty phần mềm. Kiểm thử phần mềm tự động bao gồm hai hướng tiếp cận chính là kiểm thử tĩnh và kiểm thử động. Kế thừa hai hướng tiếp cận trên, kiểm thử tượng trưng động được đề xuất với nhiều cải tiến và thu được những thành tựu nhất định. Tuy nhiên, khi áp dụng các hướng tiếp cận này, một số vấn đề nảy sinh trong quá trình sinh dữ liệu kiểm thử cho con trỏ void và con trỏ hàm. Khóa luận này đề xuất một phương pháp để tìm kiếm một tập các dữ liệu phù hợp cho các tham số con trỏ nhằm giải quyết các hạn chế kể trên. Phương pháp sinh dữ liệu kiểm thử cho con trỏ void và con trỏ hàm giúp tăng độ bao phủ kiểm thử và nâng cao độ tin cậy của mã nguồn. Ý tưởng chính của phương pháp đề xuất là tìm kiếm tất cả các kiểu phù hợp cho con trỏ void và các tham chiếu ứng viên của con trỏ hàm. Quá trình trên được thực hiện bằng cách phân tích mã nguồn trong hai phạm vi: nội bộ và ngoại bộ đơn vị được kiểm thử. Các kiểu và tham chiếu trên được sử dụng để hỗ trợ sinh các dữ liệu khởi tạo theo hướng tiếp cận của kiểm thử tượng trưng động. Phương pháp đề xuất đã được cài đặt thành một công cụ hỗ trợ kiểm thử cho nhiều thư viện và dự án nhúng viết bằng C/C++. Kết quả thực nghiệm cho thấy rằng phương pháp có khả năng sinh ra một lượng ít hơn dữ liệu kiểm thử nhưng vẫn đảm bảo nâng cao độ phủ mã nguồn đạt được so với các phương pháp truyền thống. Phương pháp đề xuất đã được tích hợp vào công cụ AKAUTAUTO - sản phẩm hợp tác nghiên cứu giữa Phòng thí nghiệm đảm bảo chất lượng phần mềm (khoa Công nghệ thông tin, Trường Đại học Công nghệ, ĐHQGHN) và đơn vị FGA (FPT Global Automative) để kiểm thử tự động các phần mềm điều khiển xe ô tô. Những kết quả tích cực của phương pháp đề xuất cho thấy tiềm năng ứng dụng thực tế để kiểm thử các thư viện và dự án nhúng viết bằng C/C++ có kích thước lớn.

\vspace{-0.5cm}
\begin{flushleft}
  \textit{\textbf{Từ khóa: } kiểm thử tự động, kiểm thử tượng trưng động, thực thi giá trị tượng trưng, kiểm thử dòng điều khiển, con trỏ void, con trỏ hàm, bộ giải SMT}
\end{flushleft}

\newpage
\addcontentsline{toc}{chapter}{Abstract}
\begin{center}
    \textbf{ABSTRACT}
\end{center}
\changefontsizes[16pt]{12pt}
\textit{\textbf{Abstract: }} 
Unit testing for C/C++ projects, especially libraries and embedded projects has been known as a complex problem. In order to improve the software reliability, automated software testing is widely applied in both research and industrial communities. Automated software testing includes two main approaches called static testing and concolic testing. Inheriting the above two approaches, concolic testing is proposed with several improvements and certain achievements. However, while applying these approaches, it is noticed that there are some issues in generating test data for void pointers and function pointers. This thesis proposed a method to generate test data for void pointers and function pointers in order to improve the testing coverage and the software reliability. The key idea of the proposed method applying concolic testing is to analyze the given source code to find all possible types for void pointers and references for function pointers. The analysis is performed on source code both inside and outside of the unit under test. These types and references are used to generate the initial test data for concolic testing method. The proposed method is implemented in a support tool to test on various C/C++ libraries and embedded projects. The experimental results show that the method significantly improves the coverage of the generated test data. The proposed method is capable of employing a limited number of test data but remaining highly competitive in comparison with other existing methods. The proposed method has been applied into the AKAUTAUTO - a research collaboration product between Software Quality Assurance Laboratory (Faculty of Information Technology, University of Engineering and Technology, VNU) and FGA (FPT Global Automative) for automated testing for automotive software. The design of the proposed method allows various C/C+ libraries and embedded projects to be tested on a large scale.

\vspace{-0.5cm}
\begin{flushleft}
  \textit{\textbf{Keywords: } Automated test data generation, concolic testing, symbolic execution, control flow testing, void pointers, function pointers, SMT Solver}
\end{flushleft}
\changefontsizes[16pt]{13pt}
