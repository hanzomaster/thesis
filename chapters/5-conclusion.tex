\chapter*{Kết luận}\label{chap5}
\addcontentsline{toc}{chapter}{Kết luận}
Khóa luận này đã trình bày chi tiết về quá trình nghiên cứu, thiết kế, và triển khai một hệ thống quản lý nhà hàng toàn diện, đáp ứng nhu cầu của cả chủ nhà hàng và khách hàng.
Hệ thống không chỉ cung cấp các công cụ quản lý hiệu quả cho nhà hàng, quán ăn mà còn tạo ra một nền tảng tương tác thuận tiện cho khách hàng thông qua các tính năng đặt bàn trực tuyến và đặt món trực tuyến thông qua quét mã QR và quản lý thông tin cá nhân.

Việc áp dụng kiến trúc vi dịch vụ cùng với việc tận dụng khả năng mở rộng tự động của GKE đã giúp hệ thống đạt được tính linh hoạt, khả năng chịu lỗi cao và khả năng đáp ứng nhu cầu sử dụng biến động. Các bộ kiểm thử đã chứng minh hiệu quả hoạt động của hệ thống trong việc tự động mở rộng và duy trì tính sẵn sàng cao, đảm bảo trải nghiệm người dùng mượt mà và không bị gián đoạn.

Tuy còn một vài hạn chế trong thiết kế và tích hợp hoàn chỉnh các luồng nghiệp vụ của hệ thống, sau khi các vấn đề được khắc phục trong tương lai gần, hệ thống sẽ tiếp tục được mở rộng và phát triển với các tính năng mới nhằm nâng cao trải nghiệm người dùng và đáp ứng tốt hơn nhu cầu thị trường.
Một số các chức năng trong số đó bao gồm tính năng đặt món trực tuyến trên trang giới thiệu của nhà hàng, quán ăn, quản lý chế độ dinh dưỡng của người dùng, gợi ý món ăn tại trang đặt món, phát triển ứng dụng di động, v.v.

Bên cạnh đó, ứng dụng hoàn toàn có tiềm năng tận dụng những dữ liệu từ cửa hàng cũng như là khách hàng sử dụng nền tảng để áp dụng công nghệ trí tuệ nhân tạo (AI) nhằm phát triển các chức năng đột phá mới.
Ví dụ, AI có thể được sử dụng để phân tích dữ liệu của người dùng, từ đó đưa ra các gợi ý món ăn tùy biến theo các thông tin thu thập được.
Hoặc AI giúp dự đoán xu hướng ẩm thực và hỗ trợ nhà hàng trong việc xây dựng chiến lược kinh doanh hiệu quả.

Với những định hướng phát triển này, hệ thống quản lý nhà hàng được kỳ vọng sẽ đóng góp tích cực vào sự phát triển của ngành dịch vụ ăn uống tại Việt Nam, giúp các nhà hàng tối ưu hóa quy trình quản lý, nâng cao chất lượng dịch vụ và tăng cường khả năng cạnh tranh trên thị trường.
Với sự phát triển không ngừng của công nghệ, hệ thống này sẽ tiếp tục được cải tiến và hoàn thiện, mang đến những giải pháp tiên tiến và hiệu quả hơn cho ngành công nghiệp này.