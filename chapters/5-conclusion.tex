\chapter{Kết luận}\label{chap5}
Khóa luận đã mô tả bối cảnh cụ thể về hai hạn chế và sự cần thiết trong việc phát triển một phương pháp kiểm thử hiệu quả khi kiểm thử đơn vị tự động cho mã nguồn C/C++ chứa hàm thiếu định nghĩa. Trong đó hạn chế đầu tiên liên quan đến quá trình chuẩn bị môi trường kiểm thử tự động bởi mã nguồn có thể chứa hàm thiếu định nghĩa gây một số lỗi như lỗi thiếu định nghĩa hàm hoặc lỗi thiếu bảng ký hiệu ảo. Hạn chế này khiến các công cụ kiểm thử tự động không thể chạy các ca kiểm thử trên mã nguồn và tiêu tốn nhiều thời gian để xử lý thủ công. Khóa luận cũng đề cập hạn chế thứ hai liên quan đến việc sinh dữ liệu kiểm thử tự động và sinh stub tự động cho mã nguồn chứa lời gọi phương thức. 

Khóa luận đã giới thiệu một phương pháp kiểm thử đơn vị tự động cho mã nguồn C/C++ chứa hàm thiếu định nghĩa để giải quyết hai hạn chế trên. Thứ nhất, khóa luận bổ sung pha xử lý hàm thiếu định nghĩa so với phương pháp truyền thống để giải quyết hạn chế thứ nhất. Việc xử lý hàm thiếu định nghĩa được chia thành hai phần: xử lý nguyên mẫu hàm ảo và xử lý nguyên mẫu hàm bình thường. Trong đó, khóa luận đã sử dụng các công cụ tiện ích của trình biên dịch để thu thập danh sách nguyên mẫu hàm thiếu định nghĩa cần quan tâm, sau đó lọc tìm ứng viên trên đồ thị cấu trúc mã nguồn và sinh thân hàm giả cho chúng. Đối với nguyên mẫu hàm ảo thiếu định nghĩa, khóa luận kết hợp tìm kiếm trên đồ thị cấu trúc mã nguồn và thuật toán lọc tìm ứng viên để xử lý các hàm ảo gây lỗi thiếu bảng ký hiệu ảo. Thứ hai, khóa luận đã cải tiến phương pháp sinh stub tự động bằng phương pháp xử lý lời gọi phương thức. Ý tưởng của phương pháp dựa trên phương pháp AS4UT nhưng bổ sung thêm bước sinh đối tượng giả cho các lời gọi phương thức. Sau đó, phương pháp đề xuất sử dụng phương pháp kiểm thử tượng trưng động để tạo ra các bộ dữ liệu kiểm thử mới.

Khóa luận đã cài đặt và tích hợp phương pháp đề xuất vào phiên bản 5.9.2-thesis của công cụ AKAUTAUTO để tiến hành một số thực nghiệm đánh giá tính hiệu quả của phương pháp đề xuất. Kết quả thực nghiệm cho thấy phương pháp đề xuất có khả năng rút ngắn đáng kể thời gian chuẩn bị môi trường kiểm thử tự động cho mã nguồn thiếu định nghĩa. Kết quả cũng cho thấy rằng phương pháp đề xuất có khả năng sinh dữ liệu kiểm thử đạt độ phủ cao hơn so với phương pháp truyền thống, đặc biệt trên các dự án có nhiều lời gọi phương thức. Tuy rằng có sự chênh lệch giữa thời gian và bộ nhớ sử dụng để sinh dữ liệu kiểm thử giữa hai phương pháp nhưng thực nghiệm cho thấy sự chênh lệnh này chấp nhận được khi áp dụng thực tế. 

Mặc dù phương pháp đề xuất đã đạt được một số kết quả đáng quan tâm như trên, phương pháp đề xuất vẫn còn một số nhược điểm cần hoàn thiện trong tương lai. Trước hết, phương pháp hiện tại chưa hỗ trợ hoàn toàn quy trình bóc tách mô-đun và xử lý hàm thiếu định nghĩa. Để áp dụng được phương pháp đề xuất, khóa luận cần kiểm thử viên bóc tách mô-đun sao cho biên dịch được. Xét về khả năng sinh dữ liệu kiểm thử tự động, phương pháp đề xuất chưa cải thiện được độ phủ khi kiểm thử tự động cho các đơn vị chứa ít lời gọi phương thức. Ngoài ra, phương pháp hiện tại chưa giải quyết một số nhược điểm của phương pháp kiểm thử tượng trưng động. Bên cạnh đó, phương pháp đề xuất cần được áp dụng trên các dự án C/C++ lớn hơn.

Nhìn chung, tuy phương pháp đề xuất còn một số hạn chế, không thể phủ nhận rằng phương pháp đã có kết quả tích cực và bộc lộ tiềm năng áp dụng thực tiễn trong việc kiểm thử đơn vị tự động các dự án C/C++. Phương pháp đề xuất giải quyết được các hạn chế phát sinh bởi mã nguồn chứa hàm thiếu định nghĩa. Qua đó, phương pháp giúp tự động hóa quá trình kiểm thử đơn vị các dự án C/C++ với chi phí tiết kiệm hơn.