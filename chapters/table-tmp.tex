Bảng \ref{table:inputFunctions} liệt kê các hàm có trong mã nguồn được lấy làm ví dụ. Bảng có hai cột tương ứng với nội dung hàm và tập ca kiểm thử sinh tự động.

\begin{longtable}{|p{0.7cm}|p{10cm}|p{4cm}|}
    % \centering
    \caption{Bảng danh sách các hàm có trong mã nguồn ví dụ}
    % \vspace{0.5cm}
    % \begin{adjustbox}{width=1\textwidth}
    % \small
    % \begin{tabular}{ |p{1cm}|p{6,5cm}|p{6,5cm}|  }
    \\
         \hline
        \textbf{STT} & \textbf{Hàm} & \textbf{Tập ca kiểm thử}\\
         \hline
         \endhead
         \hline
         \endfoot
        1 & \begin{lstlisting}[language=ES6]
export function abs(a: number) {
    if ( a > 0) return a;
    else if (a < 0) return -a;
    else return 0;
}

        \end{lstlisting}
         & \makecell[l]{a = 1, b = 0;\\ 
         a = -1, b = 0;\\
         a = 0, b = 0;}
        \\
        \hline
        2 & \begin{lstlisting}[language=ES6]
export function sum(a: number, b: number): number {
    let x = a + b;
    let y = a - b;
    let z = x + y;
    if ( z > y) {
        return a;
    } else {
        z = z + 1;
        if (z < 10) {
            return b;
        }
    }
}   
        \end{lstlisting}

        & \makecell[l]{a = 0, b = 0;\\ 
         a = 1, b = 0;\\
         a = 5, b = -5;}
        \\
        \hline
        3 & 
        \begin{lstlisting}[language=ES6]
export function simpleStringFunction3(s: string) {
    let x = s.length +1;
    let ss = "End"
    if ( x > 10) {
        if (s.startsWith("Hoaithu")) {
            return 1;
        } else if (s.includes("UET")) {
            return 2;
        } else if (s.endsWith(ss)) {
            return 3;
        }
    }
}
        \end{lstlisting}
        
        & \makecell[l]{ 
        s = "Hoaithuxxx";\\ s = "HoxUETxxxx";\\ 
         s = "Ho x@\textbackslash nxEnd";\\ s = "@xxxxxxxxnd";\\
         s = "";\\
        }
        \\
        \hline
        4 & 
        \begin{lstlisting}
export function compoundCondition(a: string, person: Person): number {
    let result = person.age + person.height;
    if (a.endsWith("hoaithu") && person.age == 10 + 2 ) {
        if (a.includes("zzz")){
        if (person.school.name.startsWith("aaaa"))
            return 1;
        }
        else return 2;
    }
    return result;
}
        \end{lstlisting}
        & \makecell[l]{
        a = "xzzzxhoaithu",\\ person = \{"age":12,\\ "height":0,\\ "school":\{  \\"name":"aaaa"  \}\};\\
        \\
        a = "xxxxxzxzzzhoaithu",\\ person = \{"age":12,\\ "height":0,\\ "school":\{  \\"name":"@xxx"  \}\};\\
        \\
         a = "hoaithu",\\ person = \{"age":12,\\ "height":0,\\ \};\\
         \\
          a = "",\\ person = \{"age":13,\\ "height":0,\\ \};\\
         }
        \\
        \hline
       5 & 
       \begin{lstlisting}
export function object_literal_expresion_test( person: Person): number {
    let names = [{name: "UET", age: 23, school: {name: "DHCN"}}];
    let list = [1,2,3,4,5];
    list[2] = 10;
    if (person.name === names[0].name && person.pets[0].getSound() == "GauGau" && person.pets[0].getChildren()[0].getSound() == "MeoMeo") {
        if (person.school.name === names[0].school.name && names[0].name.length < list[2]) {
            return 1;
        }
        else return 2;
    }
    return 1;
}
       \end{lstlisting}
        & \makecell[l]{
        person = \{"name": "UET",\\ "school":\{  \\"name":"DHCN"  \},\\ "pets":[\\ \{ "sound": "GauGau", \\ "children": [ \\ \{"sound": "MeoMeo"\}]\}]\};\\
        \\
       person = \{"name": "UET",\\ "school":\{  \\"name":""  \},\\ "pets":[\\ \{ "sound": "GauGau", \\ "children": [ \\ \{"sound": "MeoMeo"\}]\}]\};\\
        \\
        person = \{"name": "",\\ "pets":[\\ \{ "sound": "", \\ "children": [ \\ \{"sound": ""\}]\}]\};\\
         }
        \\
        \hline
       6 & Biểu thức sử dụng các phép toán với thuộc tính của tham số đối tượng   & \makecell[l]{a = person.height + person.age;\\
        if (person.height > 180)\\
        if (person.height/person.age <  8)} 
        \\ 
        \hline
       7 & 
       \begin{lstlisting}
export function external_function_test(a: number, b: number): number {
    let x: number = sum(a,b);
    let y: number = Algorithm.sum(4,5);
    let z: number = Algorithm.max(4,5);

    if (x > 3 && max(3,4) < 5) {
        if (y > 6 && z < 10) {
            if ( b >5 && a < 6 ) {
                return 1;
            }
        }
    } else return 1;
}
       \end{lstlisting}
       & \makecell[l]{s = person.getName();\\
        if (person.getName().\\startsWith(“hoaithu”))} 
        \\
         \hline
       8 & Biểu thức sử dụng các phép toán với phần tử mảng có index cụ thể &\makecell[l]{ a = arr[0] + b; \\
        a[0] = a + b + a[1];} 
        \\
         \hline
      9 & Biểu thức khai báo/gán giá trị là một mảng & a = [1,2,3,4,5,6];\\
         \hline
      10 &  Biểu thức khai báo/gán giá trị là một Json object & \makecell[l]{a= \{height: 180, age: 23, \\school:\{name: “UET”\}\}}\\
         \hline
      11 &  Biểu thức điều kiện kép bao gồm biều biểu thức điều kiện đơn thỏa mãn các tính chất 1 -> 8 & If (a > b \&\& s.length > 10 \&\& s.startsWith(“ABC”))\\
         \hline
        % \end{tabular}
    % \end{adjustbox}
    \label{table:inputFunctions}
\end{longtable}

Bảng \ref{table:expressions} liệt kê các loại biểu thức được hỗ trợ cho đến hiện tại. 

\setlength\LTleft{0pt}
\setlength\LTright{0pt}
\begin{longtable}{ |p{1cm}|p{7cm}|p{6,5cm}|}
% \begin{longtable}{ @{\extracolsep{\fill}}|p{1cm}|p{6,5cm}|p{6,5cm}|@{}}
    % \centering
    \caption{Bảng danh sách các dạng biểu thức được hỗ trợ trong mã nguồn}
    % \vspace{0.5cm}
    % \begin{adjustbox}{width=1\textwidth}
    % \small
    % \begin{tabular}{ |p{1cm}|p{6,5cm}|p{6,5cm}|  }
    \\
         \hline
         \textbf{STT} & \textbf{Kiểu biểu thức} & \textbf{Ví dụ minh họa}\\
         \hline
         \endhead
         \hline
         \endfoot
        1 & Biểu thức sử dụng các phép toán với biến nguyên thủy & \makecell[l]{ a = 1 + 2;\\
         a = 3*4/2;\\
        if (a > 1);\\ if(a > 1 + 2); \\
        if (a + 1 > 3*4)} 
        \\
        \hline
        2 & Biểu thức sử dụng các phép toán với tên biến.   & \makecell[l]{ a = b + c;\\ a = b*c;\\
        if (a > b); \\if (a > b+c); \\
        if (a + b ==  c + d); \\
        if (a*b != c/d)}
        \\
        \hline
        3 & Biểu thức sử dụng thuộc tính length của biến kiểu string & \makecell[l]{ s = “abcdef”;\\
        a = s.length;\\
        if (s.length > 10);\\
        if(s.length > a+b); \\
        if (s1.length < s2.length + a);}
        \\
        \hline
        4 & Biểu thức sử dụng phương thức startsWith(), endsWith(), includes() của biến kiểu string & \makecell[l]{ a = s.startsWith(“ABC”); \\
        a = endsWith(“DEF”);\\
        if (s.includes(“XYZ”))}
        \\
        \hline
        5 & Gán một chuỗi cụ thể cho biến kiểu string & a = “abcdef”;\\
        \hline
        6 & Biểu thức sử dụng các phép toán với thuộc tính của tham số đối tượng   & \makecell[l]{a = person.height + person.age;\\
        if (person.height > 180)\\
        if (person.height/person.age <  8)} 
        \\ 
        \hline
        7 & Biểu thức sử dụng các phép toán với các hàm getter của đối tượng & \makecell[l]{s = person.getName();\\
        if (person.getName().\\startsWith(“hoaithu”))} 
        \\
         \hline
        8 & Biểu thức sử dụng các phép toán với phần tử mảng có index cụ thể &\makecell[l]{ a = arr[0] + b; \\
        a[0] = a + b + a[1];} 
        \\
         \hline
        9 & Biểu thức khai báo/gán giá trị là một mảng & a = [1,2,3,4,5,6];\\
         \hline
        10 &Biểu thức khai báo/gán giá trị là một Json object & \makecell[l]{a= \{height: 180, age: 23, \\school:\{name: “UET”\}\}}\\
         \hline
        11 & Biểu thức điều kiện kép bao gồm biều biểu thức điều kiện đơn thỏa mãn các tính chất 1 -> 8 & If (a > b \&\& s.length > 10 \&\& s.startsWith(“ABC”))\\
         \hline
        % \end{tabular}
    % \end{adjustbox}
    \label{table:expressions}
\end{longtable}