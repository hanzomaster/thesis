%-------TITLE PAGE------%
\begin{titlepage}
	\center
	\begin{tikzpicture}[overlay,remember picture]
		\draw [line width=3pt,rounded corners=0pt,]
		($ (current page.north west) + (25mm,-25mm) $)
		rectangle
		($ (current page.south east) + (-15mm,25mm) $);
		\draw [line width=1pt,rounded corners=0pt]
		($ (current page.north west) + (26.5mm,-26.5mm) $)
		rectangle
		($ (current page.south east) + (-16.5mm,26.5mm) $);
	\end{tikzpicture}
	
	{\large \bfseries ĐẠI HỌC QUỐC GIA HÀ NỘI\\ TRƯỜNG ĐẠI HỌC CÔNG NGHỆ}\\[1cm]
	\includesvg[width=0.25\linewidth]{images/Logo_HUET.svg}\\[1cm]
	{\Large  \bfseries Trần Tuấn Thịnh}\\[1.5cm]
	{ \LARGE \bfseries  PHÁT TRIỂN HỆ THỐNG }\\[0.2cm]
    {\LARGE \bfseries HỖ TRỢ QUẢN LÝ NHÀ HÀNG VỪA VÀ NHỎ}\\[0.2cm]
     % { \LARGE \bfseries CHUYÊN NGHIỆP HÓA NGÀNH DỊCH VỤ ĂN UỐNG}\\[0.2cm]
	\hfill\\[2cm]
	{\large \bfseries KHÓA LUẬN TỐT NGHIỆP ĐẠI HỌC HỆ CHÍNH QUY}\\	
	{\large \bfseries Ngành: Công nghệ thông tin}	
	\hfill\\[5.3cm]	
	{\large \bfseries HÀ NỘI - 2024}\\	
	\vfill
\end{titlepage}

%-------TITLE PAGE+6hbk,------%
\begin{titlepage}
	\center
	\begin{tikzpicture}[overlay,remember picture]
	\draw [line width=3pt,rounded corners=0pt,]
	($ (current page.north west) + (25mm,-25mm) $)
	rectangle
	($ (current page.south east) + (-15mm,25mm) $);
	\draw [line width=1pt,rounded corners=0pt]
	($ (current page.north west) + (26.5mm,-26.5mm) $)
	rectangle
	($ (current page.south east) + (-16.5mm,26.5mm) $);
	\end{tikzpicture}
	
	{\large \bfseries ĐẠI HỌC QUỐC GIA HÀ NỘI\\ TRƯỜNG ĐẠI HỌC CÔNG NGHỆ}\\[2cm]
% 	\includegraphics[width=0.25\linewidth]{images/Logo_UET.png}\\[1cm]
	
	{\Large  \bfseries Trần Tuấn Thịnh}\\[2cm]		
	{ \LARGE \bfseries  PHÁT TRIỂN HỆ THỐNG }\\[0.2cm]
    {\LARGE \bfseries HỖ TRỢ QUẢN LÝ NHÀ HÀNG VỪA VÀ NHỎ}\\[0.2cm]
     % { \LARGE \bfseries CHUYÊN NGHIỆP HÓA NGÀNH DỊCH VỤ ĂN UỐNG}\\[0.2cm]
	\hfill\\[1.5cm]
	{\large \bfseries KHÓA LUẬN TỐT NGHIỆP ĐẠI HỌC HỆ CHÍNH QUY}\\	
	{\large \bfseries Ngành: Công nghệ thông tin}
	\hfill\\[2cm]
	\begin{flushleft}
	    	{\large \bfseries Cán bộ hướng dẫn: TS. Võ Đình Hiếu}\\
	\end{flushleft}
	\hfill\\[2.5cm]	
	\begin{flushleft}
% 	{\large \bfseries Cán bộ đồng hướng dẫn: CN. Bùi Quang Cường}\\	
	\end{flushleft}
		\hfill\\[3cm]	
	{\large \bfseries HÀ NỘI - 2024}\\		
	\vfill		
\end{titlepage}

%-------TITLE PAGE+6hbk,------%
\begin{titlepage}
	\center
	\begin{tikzpicture}[overlay,remember picture]
	\draw [line width=3pt,rounded corners=0pt,]
	($ (current page.north west) + (25mm,-25mm) $)
	rectangle
	($ (current page.south east) + (-15mm,25mm) $);
	\draw [line width=1pt,rounded corners=0pt]
	($ (current page.north west) + (26.5mm,-26.5mm) $)
	rectangle
	($ (current page.south east) + (-16.5mm,26.5mm) $);
	\end{tikzpicture}
	
	{\large \bfseries VIETNAM NATIONAL UNIVERSITY, HANOI\\ UNIVERSITY OF ENGINEERING AND TECHNOLOGY}\\[2cm]
% 	\includegraphics[width=0.25\linewidth]{images/Logo_UET.png}\\[1cm]
	
	{\Large  \bfseries Tran Tuan Thinh}\\[2cm]	
 % Develop a management system for small and medium-sized restaurants
	{ \LARGE \bfseries DEVELOP A MANAGEMENT SYSTEM }\\[0.2cm] 
	{\LARGE \bfseries FOR SMALL AND MEDIUM-SIZED EATERIES }\\[0.2cm]
    % {\LARGE \bfseries PROFESSIONALIZE THE FOOD SERVICE INDUSTRY }\\[0.2cm]
	\hfill\\[1.5cm]
	{\large \bfseries BACHELOR’S THESIS}\\	
	{\large \bfseries Major: Information Technology}
	\hfill\\[2cm]
	\begin{flushleft}
	    	{\large \bfseries Supervisor: Dr. Vo Dinh Hieu}\\
	\end{flushleft}
	\hfill\\[2.5cm]	
	\begin{flushleft}
% 	{\large \bfseries Cán bộ đồng hướng dẫn: CN. Bùi Quang Cường}\\	
	\end{flushleft}
		\hfill\\[3cm]	
	{\large \bfseries Hanoi - 2024}\\
	\vfill		
\end{titlepage}

\changefontsizes[16pt]{13pt}
\addtocontents{toc}{\vspace{-1cm}}
\addcontentsline{toc}{chapter}{Lời cam đoan}
\begin{center}
    \textbf{LỜI CAM ĐOAN}
\end{center}

% Em xin cam đoan: Khóa luận tốt nghiệp với đề tài “Phát triển hệ sinh thái hỗ trợ quán ăn và thực khách giúp chuyên nghiệp hóa ngành dịch vụ ăn uống” trong báo cáo này là của em.
% Những gì em viết ra không có sự sao chép từ các tài liệu, không sử dụng kết quả của người khác mà không trích dẫn cụ thể.
% Đây là công trình nghiên cứu cá nhân em tự phát triển, không sao chép mã nguồn của người khác.
% Nếu vi phạm những điều trên, em xin chấp nhận tất cả những truy cứu về trách nhiệm theo quy định của Trường Đại học Công nghệ - ĐHQGHN.

Tôi tên là Trần Tuấn Thịnh, sinh viên của lớp QH-2020-I/CQ-C-CLC, ngành Công nghệ thông tin hệ chất lượng cao.
Tôi xin cam đoan rằng đề tài khóa luận tốt nghiệp của tôi với chủ đề \say{Phát triển hệ thống hỗ trợ quản lý nhà hàng vừa và nhỏ} là sản phẩm nghiên cứu của bản thân tôi và chưa bao giờ được nộp làm khóa luận tốt nghiệp cho bất kỳ trường đại học nào, trong đó có Trường Đại học Công nghệ.
Tôi cam đoan rằng tôi không sao chép bất kỳ nội dung nào từ các tài liệu khác, cũng như không sử dụng kết quả của bất kỳ người nào khác mà không có trích dẫn cụ thể.
Đây là công trình nghiên cứu do cá nhân tôi thực hiện với sự hướng dẫn của TS Võ Đình Hiếu, không chứa bất kỳ mã nguồn nào được sao chép, chỉnh sửa từ cá nhân, tổ chức nào khác mà không được cho phép.
Nếu vi phạm những điều trên, tôi hoàn toàn chấp nhận mọi trách nhiệm theo quy định của Trường Đại học Công nghệ.

\begin{flushright}
	\begin{varwidth}{\linewidth}\centering
		Hà Nội, ngày 27 tháng 5 năm 2024\\
		Sinh viên\\[2cm]
		Trần Tuấn Thịnh
	\end{varwidth}
\end{flushright}

\newpage

\addcontentsline{toc}{chapter}{Lời cảm ơn}
\begin{center}
    \textbf{LỜI CẢM ƠN}
\end{center}

Lời đầu tiên cho phép em được gửi lời cảm ơn tới Khoa Công Nghệ Thông Tin – Trường Đại học Công nghệ - ĐHQG Hà Nội đã tạo điều kiện thuận lợi cho em được học tập, nghiên cứu và thực hiện đề tài tốt nghiệp này.

Em cũng xin được bày tỏ lòng biết ơn sâu sắc tới thầy Võ Đình Hiếu đã tận tình hướng dẫn, đóng góp những ý kiến xác đáng để em có thể hoàn thành khóa luận một cách tốt nhất.
Thời gian được làm việc và nghiên cứu tại phòng thí nghiệm thật sự đã đem lại những kinh nghiệm, những kiến thức quý báu và vô giá đối với bản thân em.

Cuối cùng, em cũng vô cùng biết ơn những thầy cô trong trường đã tận tình giảng dạy, trang bị cho em những kiến thức quan trọng để em có đầy đủ kiến thức và kỹ năng để có thể tự tin đi tiếp những chặng đường tiếp theo.

Chúc mọi người luôn luôn vui vẻ và gặt hái được nhiều thành công trong cuộc sống.

\newpage
\addcontentsline{toc}{chapter}{Tóm tắt}
\begin{center}
    \textbf{TÓM TẮT}
\end{center}
\changefontsizes[16pt]{12pt}
\textit{\textbf{Tóm tắt: }} 
Ngành dịch vụ ăn uống tại Việt Nam đang trên đà phát triển mạnh mẽ cùng với sự bùng nổ của công nghệ, kèm theo đó là khẩu vị thực khách ngày càng trở nên tinh tế và đa dạng, buộc các nhà hàng và quán ăn phải không ngừng đổi mới và sáng tạo để bắt kịp xu hướng.
Ngoài việc phục vụ khách đến ăn trực tiếp, các quán ăn giờ đây còn tham gia vào các nền tảng đặt đồ ăn trực tuyến để quảng bá và thu hút thêm khách mua hàng giúp mở rộng thị trường.
Tuy nhiên, chi phí tham gia các nền tảng này dao động từ 20 đến 30\% doanh thu của quán, đây là một thách thức lớn cho sự phát triển của các quán ăn có quy mô vừa và nhỏ.
Nhằm giải quyết vấn đề này, bài luận sẽ trình bày về quá trình phát triển một nền tảng tập trung hỗ trợ quản lý nhà hàng và hỗ trợ đặt đồ ăn dành riêng cho các nhà hàng vừa và nhỏ.

Hệ thống này sẽ cung cấp một ứng dụng quản lý cho một doanh nghiệp với các chức năng đầy đủ và toàn diện như quản lý bán hàng, doanh thu, nhân viên, khách hàng cũng như hỗ trợ quảng bá cho nhà hàng.
Điều này giúp tối ưu hóa hoạt động kinh doanh của nhà hàng, giảm thiểu chi phí nhân viên và vận hành, quản lý.
Ngoài ra, hệ thống cũng tích hợp tính năng đặt đồ ăn trực tuyến với giao diện thân thiện, dễ sử dụng, cho phép khách hàng dễ dàng đặt món và thanh toán tại nhà hàng trên nền tảng.

Bằng việc áp dụng kiến trúc vi dịch vụ (microservice) khi phát triển và triển khai hệ thống, từng dịch vụ như thông báo, đặt đồ ăn, quản lý nhà hàng, v.v. sẽ được tách biệt riêng với nhau giúp tăng tính sẵn sàng cũng như là khả năng chịu lỗi của hệ thống.
Ngoài ra về sau khi số lượng người dùng tăng, kiến trúc này cũng đảm bảo khả năng nâng cấp linh hoạt, dễ dàng và đặc biệt trong những khung giờ cao điểm khi số lượng người dùng vượt quá mức sử dụng thường thấy, hệ thống hoàn toàn có thể tự động mở rộng mà không cần tới sự can thiệp của quản trị viên hệ thống.
\vspace{-0.5cm}
\begin{flushleft}
  \textit{\textbf{Từ khóa: } Microservice, tự động mở rộng, quản lý quán ăn, nhà hàng, nền tảng tập trung}
\end{flushleft}

\newpage
\addcontentsline{toc}{chapter}{Abstract}
\begin{center}
    \textbf{ABSTRACT}
\end{center}
\changefontsizes[16pt]{12pt}
\textit{\textbf{Abstract: }} 
The Vietnamese foodservice industry is experiencing rapid growth alongside the explosion of technology.
With increasingly sophisticated and diverse customer palates, restaurants and eateries are constantly innovating and adapting to keep up with trends.
In addition to serving dine-in customers, eateries are now also participating in online food ordering platforms to promote themselves and attract more customers, expanding their market reach.
However, the cost of participating in these platforms ranges from 20 to 30\% of the restaurant's revenue, posing a significant challenge for the growth of small and medium-sized eateries.
To address this issue, this paper presents the development process of a centralized system to support restaurant management and online food ordering specifically for small and medium-sized restaurants.

The system will provide a management application for businesses with comprehensive and full-fledged functionalities such as sales, revenue, staff, and customer management, as well as restaurant promotion support.
This will help optimize restaurant operations, minimize staff and operational costs, and management.
In addition, the system also integrates an online food ordering feature with a friendly and easy-to-use interface, allowing customers to easily order and pay for food at the restaurant on the platform.

By applying a microservice architecture to the development and deployment of the system, each service such as notifications, food ordering, restaurant management, etc., will be separated from each other, increasing the availability and fault tolerance of the system.
Additionally, as the number of users increases, this architecture also ensures flexible and easy upgradeability, and especially during peak hours when the number of users exceeds the usual usage, the system can automatically scale without the need for system administrator intervention.
\vspace{-0.5cm}
\begin{flushleft}
  \textit{\textbf{Keywords: } Microservice, auto-scaling, restaurant management, restaurants, centralized platform}
\end{flushleft}
\changefontsizes[16pt]{13pt}
